% Atur variabel berikut sesuai namanya

% nama
\newcommand{\name}{Dzakirozaan Uzlahwasata}
\newcommand{\authorname}{Uzlahwasata, Dzakirozaan}
\newcommand{\nickname}{Dzakirozaan}
\newcommand{\advisor}{Annisaa Sri Indrawanti, S. Kom., M. Kom.}
\newcommand{\coadvisor}{Fuad Dary Rosyadi, S.Kom., M.Kom.}
\newcommand{\examinerone}{Dr. Galileo Galilei, S.T., M.Sc}
\newcommand{\examinertwo}{Friedrich Nietzsche, S.T., M.Sc}
\newcommand{\headofdepartment}{Dr.tech.Ir. Raden Venantius Hari Ginardi, M.Sc}

% identitas
\newcommand{\nrp}{5027211066}
\newcommand{\advisornip}{19910523 202012 2 018}
\newcommand{\coadvisornip}{19960910 202406 1 003}
\newcommand{\examineronenip}{18560710 194301 1 001}
\newcommand{\examinertwonip}{18560710 194301 1 001}
\newcommand{\examinerthreenip}{18560710 194301 1 001}
\newcommand{\headofdepartmentnip}{19650518 199203 1 003}

% judul
\newcommand{\tatitle}{OPTIMASI \textit{TASK SCHEDULING} MENGGUNAKAN \textit{ARTIFICAL BEE COLONY} (ABC) DAN \textit{ELITE OPPOSITION-BASED LEARNING} (EOBL) PADA \textit{CLOUD ENVIRONMENT}}
\newcommand{\engtatitle}{\emph{TASK SCHEDULING OPTIMIZATION USING ARTIFICIAL BEE COLONY (ABC) AND ELITE OPPOSITION-BASED LEARNING (EOBL) IN CLOUD ENVIRONMENT}}

% tempat
\newcommand{\place}{Surabaya}

% jurusan
\newcommand{\studyprogram}{Teknologi Informasi}
\newcommand{\engstudyprogram}{Information Technology}

% fakultas
\newcommand{\faculty}{Teknologi Elektro dan Informatika Cerdas}
\newcommand{\engfaculty}{Intelligent Electrical and Informatics Technology}

% singkatan fakultas
\newcommand{\facultyshort}{FTEIC}
\newcommand{\engfacultyshort}{ELECTICS}

% departemen
\newcommand{\department}{Teknologi Informasi}
\newcommand{\engdepartment}{Information Technology}

% kode mata kuliah
\newcommand{\coursecode}{ET234801}
