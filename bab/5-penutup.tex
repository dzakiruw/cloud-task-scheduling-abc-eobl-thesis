\chapter{PENUTUP}

\section{Kesimpulan}
Berdasarkan hasil pengujian yang telah dilakukan dari berbagai skenario, dapat disimpulkan sebagai berikut:
\begin{enumerate} [nolistsep]
\item Penerapan algoritma \textit{Artificial Bee Colony} (ABC) yang dikombinasikan dengan \textit{Elite Opposition-Based Learning} (EOBL) dimulai dengan inisialisasi populasi dan evaluasi \textit{fitness} setiap individu. Berdasarkan probabilitas acak (\textit{Pr}), sistem memilih apakah akan melanjutkan dengan prosedur ABC atau menggunakan EOBL untuk menghasilkan solusi berbasis oposisi \textit{elite}. Jika probabilitas acak lebih kecil dari ambang batas (\textit{Pe}), solusi \textit{elite} dipilih untuk memperluas ruang pencarian, memberikan variasi dalam populasi, dan memungkinkan pencarian solusi yang lebih optimal. Fase-fase lebah pekerja, pengamat, dan penjelajah dari ABC tetap berjalan, dengan tujuan meningkatkan eksplorasi dan pencarian solusi optimal selama iterasi.

\item Penerapan EOBL pada algoritma ABC membantu menghindari jebakan lokal, mempercepat konvergensi, dan meningkatkan pengalokasian sumber daya. Secara keseluruhan, ABC-EOBL lebih efisien 18,40\% dibandingkan dengan algoritma lainnya pada semua skenario yang diuji pada parameter \textit{imbalance degree}, menunjukkan distribusi sumber daya yang lebih seimbang dan efisien. Hal ini membuktikan bahwa ABC-EOBL memiliki keunggulan dalam hal penjadwalan tugas, terutama dalam menjaga keseimbangan alokasi sumber daya. Namun, meskipun ada peningkatan dalam distribusi sumber daya, penerapan EOBL juga menyebabkan penurunan kinerja pada beberapa parameter lainnya, seperti \textit{makespan}, \textit{average start time}, \textit{average finish time}, \textit{average execution time}, \textit{average waiting time}, \textit{scheduling length}, \textit{throughput}, \textit{resource utilization}, dan \textit{energy consumption}, yang disebabkan oleh peningkatan kompleksitas dalam pencarian solusi yang lebih beragam.
\end{enumerate}

\section{Saran}
Berdasarkan hasil temuan dan keterbatasan yang ada, beberapa saran pengembangan yang dapat dilakukan untuk penelitian selanjutnya adalah:
\begin{enumerate} [nolistsep]
\item Pengujian algoritma ABC-EOBL pada \textit{dataset} yang lebih besar dan beragam, untuk memberikan gambaran yang lebih komprehensif mengenai kinerja algoritma dalam menangani variasi data dan skala yang lebih kompleks, sesuai dengan penerapan \textit{cloud environment}.
\item Mengingat penurunan kinerja pada beberapa parameter, seperti \textit{makespan}, \textit{average start time,} \textit{average finish time}, \textit{average waiting time}, \textit{average execution time}, \textit{throughput}, \textit{scheduling length}, \textit{resource utilization}, dan \textit{energy consumption}, penelitian selanjutnya dapat mengeksplorasi modifikasi algoritma ABC-EOBL untuk memperbaiki kinerja pada parameter-parameter tersebut. Salah satu pendekatannya adalah dengan mengurangi kompleksitas pencarian solusi untuk meningkatkan efisiensi pada parameter waktu. Selain itu, untuk evaluasi yang lebih komprehensif, penelitian selanjutnya dapat mempertimbangkan pengukuran kinerja menggunakan parameter tambahan sesuai studi yang ada.
\item Eksperimen dengan kombinasi ABC-EOBL dengan algoritma lain dalam \textit{Swarm Intelligence} atau algoritma \textit{metaheuristik} lainnya, seperti \textit{Ant Colony Optimization} (ACO) atau \textit{Grey Wolf Optimization} (GWO), guna memperbaiki aspek-aspek yang kurang optimal, seperti waktu eksekusi, serta meningkatkan kinerja penjadwalan tugas di \textit{cloud environment}.
\item Pengujian dengan skala yang lebih besar dan lebih representatif dari \textit{cloud environment} nyata untuk memastikan algoritma yang dikembangkan dapat diterapkan secara efektif pada sistem \textit{cloud} yang lebih dinamis dan kompleks, sesuai dengan tujuan pengembangan algoritma yang aplikatif.
\end{enumerate}