\begin{center}
  \Large
  \textbf{KATA PENGANTAR}
\end{center}

\addcontentsline{toc}{chapter}{KATA PENGANTAR}

\vspace{2ex}

% Ubah paragraf-paragraf berikut dengan isi dari kata pengantar

Puji dan syukur penulis panjatkan kehadirat Tuhan Yang Maha Esa, atas segala rahmat, karunia, dan hidayah-Nya, sehingga penulis dapat menyelesaikan penyusunan buku Tugas Akhir yang berjudul "OPTIMASI \textit{TASK SCHEDULING} MENGGUNAKAN \textit{ARTIFICIAL BEE COLONY} (ABC) DAN \textit{ELITE OPPOSITION-BASED LEARNING} (EOBL) PADA \textit{CLOUD ENVIRONMENT}" dengan baik dan tepat waktu. Tugas Akhir ini disusun dengan harapan dapat memberikan kontribusi terhadap pengembangan ilmu pengetahuan, khususnya di bidang teknologi informasi dan komputasi awan.

Penulis menyadari bahwa selama proses penyusunan Tugas Akhir ini, banyak pihak yang telah memberikan bimbingan, dukungan, doa, dan semangat. Oleh karena itu, dengan segala kerendahan hati, penulis mengucapkan terima kasih yang sebesar-besarnya kepada:

\begin{enumerate}[nolistsep]

  \item Kepada Bapak, Ibu, serta Adik saya, yang senantiasa memberikan doa yang tak henti, dukungan moril dan materiel, serta kasih sayang yang menjadi sumber kekuatan bagi penulis selama menempuh pendidikan hingga penyelesaian Tugas Akhir ini.

  \item Ibu Annisaa Sri Indrawanti, S. Kom., M. Kom., selaku Dosen Pembimbing I dan Bapak Fuad Dary Rosyadi, S.Kom., M.Kom., selaku Dosen Pembimbng II yang dengan kesabaran telah meluangkan waktu, tenaga, dan pikiran untuk memberikan bimbingan, arahan, serta masukan sejak awal hingga akhir penyusunan Tugas Akhir ini.

  \item Aisyah Ayu Shafira Putri, atas rasa sayang, cinta, perhatian yang tulus, dan motivasi konstruktif yang senantiasa menyertai perjalanan penulis selama masa perkuliahan.

  \item Rendy, Wisnuyasha, Rangga Aldo, Azril, dan Rifki, selaku rekan seperjuangan yang selalu ada sejak awal perkuliahan, menemani dan membantu dalam setiap prosesnya.

  \item Rafly, Yoga, Wahyu, Alex, Aloy, Wirid, Kevin, Sulthan, Ahnaf, Reynold, Gunggus, Yasa, Hilmy, Defa, dan Zacky, selaku rekan Karyo dan Tebong yang senantiasa membantu, mewarnai, dan meramaikan kehidupan penulis dalam masa perkuliahan.

  \item Arfan, Abdul Zaki, Ilham, Frans, Reyhan, Aqil, Diba, Elvara, dan Della, selaku rekan seperjuangan pengerjaan Tugas Akhir yang telah memberikan warna semester akhir penulis.

  \item Teman-teman MR Bean, P Wacana, BEM FTEIC ITS, Kaisar Khmer, Venezuela, \textit{Intern}, dan Etrea, yang telah membentuk penulis menjadi seperti saat ini. 
    
  \item Semua pihak yang tidak dapat penulis sebutkan satu per satu, yang telah memberikan bantuan dan dukungan dalam penyelesaian Tugas Akhir ini.

\end{enumerate}

Penulis menyadari bahwa Tugas Akhir ini masih jauh dari kata sempurna karena keterbatasan pengetahuan dan pengalaman. Oleh karena itu, penulis sangat mengharapkan kritik dan saran yang membangun dari para pembaca demi kesempurnaan di masa mendatang. Akhir kata, semoga Tugas Akhir ini dapat memberikan manfaat dan menambah ilmu pengetahuan.

\begin{flushright}
  \begin{tabular}[b]{c}
    \place{}, \MONTH{} \the\year{} \\
    \\
    \\
    \\
    \\
    \name{}
  \end{tabular}
\end{flushright}
