\begin{center}
  \large\textbf{ABSTRAK}
\end{center}

\addcontentsline{toc}{chapter}{ABSTRAK}

\vspace{2ex}

\begingroup
% Menghilangkan padding
\setlength{\tabcolsep}{0pt}

\noindent
\begin{tabularx}{\textwidth}{l >{\centering}m{2em} X}
  Nama Mahasiswa    & : & \name{}         \\

  Judul Tugas Akhir & : & \tatitle{}      \\

  Pembimbing        & : & 1. \advisor{}   \\
                    &   & 2. \coadvisor{} \\
\end{tabularx}
\endgroup

% Ubah paragraf berikut dengan abstrak dari tugas akhir
Penjadwalan tugas (\textit{task scheduling}) yang efisien pada lingkungan \textit{cloud computing} merupakan tantangan krusial untuk memaksimalkan penggunaan sumber daya. Algoritma \textit{Artificial Bee Colony} (ABC) yang umum digunakan sering kali mengalami kelemahan dalam hal pemanfaatan sumber daya dan cenderung terjebak pada solusi lokal. Untuk mengatasi masalah ini, penelitian ini mengusulkan optimasi melalui kombinasi algoritma ABC dengan pendekatan \textit{Elite Opposition-Based Learning} (EOBL). Metode ini diimplementasikan dengan strategi oposisi berdasarkan solusi elite guna memperluas ruang pencarian dan menghindari konvergensi prematur. Hasil pengujian menunjukkan bahwa pendekatan ABC-EOBL secara signifikan meningkatkan keseimbangan alokasi sumber daya, terbukti dari performa parameter \textit{imbalance degree} yang 18,40\% lebih efisien dibandingkan algoritma pembanding. Meskipun unggul dalam parameter \textit{imbalance degree}, penerapan EOBL menyebabkan penurunan kinerja pada metrik lain seperti \textit{makespan}, \textit{average start time}, \textit{average finish time}, \textit{average execution time}, \textit{average waiting time}, \textit{scheduling length}, \textit{throughput}, \textit{resource utilization}, dan \textit{energy consumption}, akibat meningkatnya kompleksitas pencarian. Kesimpulannya, metode ABC-EOBL menawarkan keuntungan signifikan untuk distribusi sumber daya yang seimbang, namun penggunaannya harus mempertimbangkan adanya \textit{trade-off} dengan efisiensi waktu eksekusi.

% Ubah kata-kata berikut dengan kata kunci dari tugas akhir
Kata Kunci: Penjadwalan Tugas, Komputasi Awan, \textit{Artificial Bee Colony}, \textit{Elite Opposition-Based Learning}.