\begin{center}
  \large\textbf{ABSTRACT}
\end{center}

\addcontentsline{toc}{chapter}{ABSTRACT}

\vspace{2ex}

\begingroup
% Menghilangkan padding
\setlength{\tabcolsep}{0pt}

\noindent
\begin{tabularx}{\textwidth}{l >{\centering}m{3em} X}
  \emph{Name}     & : & \name{}         \\

  \emph{Title}    & : & \engtatitle{}   \\

  \emph{Advisors} & : & 1. \advisor{}   \\
                  &   & 2. \coadvisor{} \\
\end{tabularx}
\endgroup

% Ubah paragraf berikut dengan abstrak dari tugas akhir dalam Bahasa Inggris
\textit{Efficient task scheduling in cloud computing environments is a crucial challenge for maximizing resource utilization. The commonly used Artificial Bee Colony (ABC) algorithm often suffers from weaknesses in resource utilization and a tendency to get trapped in local optima. To address this issue, this research proposes an optimization by combining the ABC algorithm with the Elite Opposition-Based Learning (EOBL) approach. This method is implemented using an opposition strategy based on elite solutions to expand the search space and avoid premature convergence. Experimental results show that the ABC-EOBL approach significantly improves the balance of resource allocation, evidenced by an 18.40\% more efficient performance in the imbalance degree parameter compared to benchmark algorithms. Despite its superiority in the imbalance degree parameter, the implementation of EOBL leads to a decrease in performance across other metrics such as makespan, average start time, average finish time, average execution time, average waiting time, scheduling length, throughput, resource utilization, and energy consumption, due to increased search complexity. In conclusion, the ABC-EOBL method offers a significant advantage for balanced resource distribution, but its application must consider the trade-off with execution time efficiency.}

% Ubah kata-kata berikut dengan kata kunci dari tugas akhir dalam Bahasa Inggris
\emph{Keywords}: \textit{Task Scheduling}, \textit{Cloud Computing}, \textit{Artificial Bee Colony}, \textit{Elite Opposition-Based Learning}.
